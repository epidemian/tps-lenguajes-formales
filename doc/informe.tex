% Tamaño de letra.
\documentclass[12pt,titlepage]{article}
%\documentclass{article}

%------------------------------ Paquetes ----------------------------------

% Paquetes:

\usepackage[T1]{fontenc}

%Para comentarios multilínea.
\usepackage{verbatim}

% Para tener cabecera y pie de página con un estilo personalizado.
\usepackage{fancyhdr}

% Codificación UTF-8
\usepackage[utf8]{inputenc}

% Tipografía
\usepackage{palatino} % Esta es genial!
\linespread{1.05} % Palatino queda mejor con un poco más de interlineado.
%\usepackage{times} % Times New Roman.


% Castellano.
\usepackage[spanish]{babel}

% Tamaño de página y márgenes.
\usepackage[a4paper,left=2cm,right=2cm,headheight=16pt]{geometry}
%\usepackage[left=2cm,top=3cm,right=2cm,bottom=1cm,head=1.5cm,includefoot]{geometry}

% Para \href y \url
\usepackage[pdfauthor   ={Demian Ferreiro}%
           ,pdftitle    ={Trabajos Prácticos - Lenguajes Formales}%
]{hyperref}
\hypersetup{
colorlinks=true,
linkcolor=black,
pdfborder= 0 0 0
}

% Gráficos:

% Para generar pdf.
\usepackage[pdftex]{graphicx}
\usepackage{float}
\usepackage{pdfpages}

% Para captions.
\usepackage{caption}

% Sarpadísima tipografía para listings y \texttt
\usepackage{inconsolata}

\usepackage{xcolor}

% Para ejemplos de código.
\usepackage{listings}
\lstset{language=Lisp
  ,basicstyle=\ttfamily\small % Usa inconsolata
  ,identifierstyle=\ttfamily
  ,keywordstyle=\color[rgb]{0,0,1}
  ,commentstyle=\color[rgb]{0.133,0.545,0.133}
  ,stringstyle=\color[rgb]{0.627,0.126,0.941}
  ,morekeywords={true,false}
  ,showstringspaces=false     % No muestra underscores en los espacios de los strings.
  ,backgroundcolor=\color{blue!10} % Un color suave de fondo (menos chocante que el frame=single)
  ,breaklines=false            % Wrappea las lineas automáticamente.
  ,belowskip=0pt               % Reduce el espacio entre un listing y el párrafo siguiente
  ,inputencoding=utf8
  ,literate={á}{{\'a}}1
           {é}{{\'e}}1
           {í}{{\'i}}1
           {ó}{{\'o}}1
           {ú}{{\'u}}1
  %frame=single               % Un recuadro en los listings.
}

% Para fórmulas matemáticas.
\usepackage{amssymb,amsmath}

% Para pseudocódigo. Instalar texlive-science.
\usepackage{algorithmic}
\usepackage{algorithm}

% Para poder hacer \begin{Verbatim}[samepage=true]
\usepackage{fancyvrb}


% Son necesarios?
%\usepackage{float}

%------------------------------ ~paquetes ---------------------------------

%------------------------- Inicio del documento ---------------------------

\begin{document}

% ---------------------- Encabezado y pie de página -----------------------

% Encabezado: sección a la derecha.
% Pie de página: número de página a la derecha.

\pagestyle{fancy}
\renewcommand{\sectionmark}[1]{\markboth{}{\thesection\ \ #1}}
\lhead{}
\chead{}
\rhead{\rightmark}
\lfoot{}
\cfoot{}
\rfoot{\thepage}

% ---------------------- ~Encabezado y pie de página ----------------------

% -------------------------- Título y autor(es) ---------------------------

\title{Prolog}
\author{}

% -------------------------- ~Título y autor(es) --------------------------

% ------------------------------- Carátula --------------------------------

\begin{titlepage}

\thispagestyle{empty}

% Logo facultad.
\begin{center}
\includegraphics[scale=0.55]{./fiuba}\\
\textsc{\Large Universidad de Buenos Aires}\\[0.2cm]
\textsc{\Large Facultad de Ingeniería}\\[1.5cm]

% Título central.

\textsc{\large Lenguajes Formales (75.14)} \\[0.3cm]
% \textsc{\large Trabajos Prácticos} \\[0.5cm]

\rule{\linewidth}{0.5mm} \\[0.4cm]
{\huge \bfseries Trabajos Prácticos} \\
%{\Large \bfseries Introducción al lenguaje y a la programación lógica}
\rule{\linewidth}{0.5mm} \\[1cm]

\vfill

\Large 
\begin{tabular}{lll}
Demian Ferrerio & 88443 & epidemian@gmail.com \\[0.5cm]
\end{tabular}

% Pie de página de la carátula.
{\Large \today}

\end{center}
\end{titlepage}

% ------------------------------- ~Carátula -------------------------------

% -------------------------------- Índice ---------------------------------

% Hago que las páginas se comiencen a contar a partir de aquí.
\setcounter{page}{1}

% Índice.
\tableofcontents
\newpage

% -------------------------------- ~Índice --------------------------------

% ----------------------------- Inicio del tp -----------------------------

\clearpage	

% Saca la indentación de los párrafos y añade un espacio entre cada uno.
\setlength{\parindent}{0pt}
\setlength{\parskip}{2ex plus 0.5ex minus 0.2ex}

% Luego del índice, links con color.
\hypersetup{
linkcolor=red
}

% Cosas Nuevas -----------------------------------------------------------------


\section{GPS}
\section{Intérprete TLC}
\section{Intérprete C}
\section{N Reinas}

% ------------------------------ Fin del tp -------------------------------

\end{document}

%---------------------------- Fin del documento ---------------------------
